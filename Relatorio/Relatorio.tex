%\documentclass[11pt]{article}
\documentclass[11pt]{IEEEtran}

\usepackage[utf8]{inputenc}
\usepackage[brazilian]{babel}

\usepackage{indentfirst}
\usepackage{xcolor}         % Permite o uso de cores no texto
\usepackage{colortbl}       % Permite colorir tabelas
\usepackage{amsmath}        % Permite escrever os conjuntos matemáticos
\usepackage{amssymb}        % Permite usar mais símbolos matemáticos, como o não existe.
\usepackage{amsfonts}
\usepackage{enumerate}      % Inclui a função enumerate 
\usepackage{graphicx}
\usepackage{hyperref}       % Permite o uso de hyperlinks no texto, com um link clicável
\usepackage{tikz}           % Permite criar imagens
\usepackage{multicol}       % Permite usar texto com várias colunas

\usepackage[top=2cm, bottom=2cm, left=3cm, right=3cm]{geometry}

\renewcommand{\baselinestretch}{1.2}
\setlength\parindent{5.5mm}

\renewcommand{\b}[1]{\textbf{#1}} %Define um resumo de um comando
\newcommand{\blue}[1]{\textcolor{blue}{#1}}
\newcommand{\blueb}[1]{\textbf{\textcolor{blue}{#1}}}

\newcommand{\inn}{$\in \mathbb{N}$ \ } % Abreviação para "no conjunto dos naturais".
\newcommand{\inz}{$\in \mathbb{Z}$ \ } % Abreviação para "no conjunto dos inteiros".

\newcommand{\pontoI}{\begin{enumerate}[\indent \indent $\bullet$]} % Cria o início da indentação normal do ponto
\newcommand{\tracoI}{\begin{enumerate}[\indent -]} % Cria o início da indentação normal do traço.

\newcommand{\fim}{\end{enumerate}} %Finaliza a indentação do ponto

\newcommand{\imp}{$\Rightarrow$ \ } % Cria a seta de implicação

\title{Compartilhamento de arquivos por FTP \\ \LARGE EzShare}
\author{Gabriel Henrique do Nascimento Neres - 221029140 \\ Guilherme Nonato da Silva - 221002020}

\begin{document}

\maketitle

\begin{abstract}
   . 
\end{abstract}

\begin{keywords}
   FTP, .
\end{keywords}


% \textbf{\textit{Palavras-chave — }FTP, .} 

\section{Introdução}
% Contextualização do tema do trabalho dentro da disciplina e da área de redes de computadores. Descrição da estruturação do relatório.
% O tema do trabalho é a implementação de um cliente de um servidor FTP. A descrição da estrutura deve ser relacionada com a descrição das divisões dos capítulos e o que será tratada em cada uma



\section{Fundamentação Teórica}
% Descrição dos conceitos teóricos e técnicas que foram utilizados e aplicados para desenvolver a compreensão do trabalho e o ambiente experimental.
% Conceitos teóricos que são importantes para o projeto:
% FTP: Definição do protocolo e funcionamento do servidor
% TCP: Definir o funcionamento da comunicação para a explicação da parte do WireShark.
% Princípio da transferência de dados.

O projeto da construção de um cliente para um servidor do tipo FTP necessitou do estudo e a compreensão de alguns dados teóricos. Dentre as informações utilizadas podem ser elencadas os conceitos ligados ao protocolo utilizado por um servidor FTP, o funcionamento de uma comunicação TCP e os princípios da transferência de dados.

O \b{protocolo FTP} defini diversas instruções e e funcionalidades baseado no protocolo do TELNET. Os principais conceitos que precisaram ser conhecidos para o desenvolvimento do cliente que iria se comunicar com o servidor foram:
\pontoI
   \item \b{Caminho de dados}: Para a transferência de dados entre o servidor e o cliente é preciso estabelecer um caminho de dados entre os dois. O caminho de dados é um canal que é criado para a transferência de um único "arquivo" entre o \textit{host} e o servidor, o que cria a necessidade de iniciar e finalizar a conexão de dados entre eles para cada solicitação feita.
   \item \b{Conexão Passiva e Ativa}: O conceito foi importante para determinar qual tipo de conexão deveria ser estabelecida pelo cliente para alcançar o objetivo desejado. Esse conceito está diretamente ligado com a criação do caminho de dados.
\fim


\section{Análise de Resultados}

\subsection{\textit{WireShark}}
\section{Conclusões}
% Deve ter entre 5 e 10 linhas resumindo as conclusões obtidas com o trabalho.
 
% Verificar a formatação das referências
\begin{thebibliography}{10}
   \bibitem[1]{}{tkinter — Python interface to Tcl/Tk, \href{https://docs.python.org/3/library/tkinter.html}{https://docs.python.org/3/library/tkinter.html}}
   \bibitem[2]{}{socket — Low-level networking interface, \href{https://docs.python.org/3/library/socket.html}{https://docs.python.org/3/library/socket.html}}
\end{thebibliography}


\end{document}
